%\VignetteIndexEntry{Data day}
%\VignetteEngine{R.rsp::tex}

\documentclass[xcolor=dvipsnames, aspectratio=1610, 9pt]{beamer}\usepackage[]{graphicx}\usepackage[]{color}
%% maxwidth is the original width if it is less than linewidth
%% otherwise use linewidth (to make sure the graphics do not exceed the margin)
\makeatletter
\def\maxwidth{ %
  \ifdim\Gin@nat@width>\linewidth
    \linewidth
  \else
    \Gin@nat@width
  \fi
}
\makeatother

\definecolor{fgcolor}{rgb}{0.345, 0.345, 0.345}
\newcommand{\hlnum}[1]{\textcolor[rgb]{0.686,0.059,0.569}{#1}}%
\newcommand{\hlstr}[1]{\textcolor[rgb]{0.192,0.494,0.8}{#1}}%
\newcommand{\hlcom}[1]{\textcolor[rgb]{0.678,0.584,0.686}{\textit{#1}}}%
\newcommand{\hlopt}[1]{\textcolor[rgb]{0,0,0}{#1}}%
\newcommand{\hlstd}[1]{\textcolor[rgb]{0.345,0.345,0.345}{#1}}%
\newcommand{\hlkwa}[1]{\textcolor[rgb]{0.161,0.373,0.58}{\textbf{#1}}}%
\newcommand{\hlkwb}[1]{\textcolor[rgb]{0.69,0.353,0.396}{#1}}%
\newcommand{\hlkwc}[1]{\textcolor[rgb]{0.333,0.667,0.333}{#1}}%
\newcommand{\hlkwd}[1]{\textcolor[rgb]{0.737,0.353,0.396}{\textbf{#1}}}%
\let\hlipl\hlkwb

\usepackage{framed}
\makeatletter
\newenvironment{kframe}{%
 \def\at@end@of@kframe{}%
 \ifinner\ifhmode%
  \def\at@end@of@kframe{\end{minipage}}%
  \begin{minipage}{\columnwidth}%
 \fi\fi%
 \def\FrameCommand##1{\hskip\@totalleftmargin \hskip-\fboxsep
 \colorbox{shadecolor}{##1}\hskip-\fboxsep
     % There is no \\@totalrightmargin, so:
     \hskip-\linewidth \hskip-\@totalleftmargin \hskip\columnwidth}%
 \MakeFramed {\advance\hsize-\width
   \@totalleftmargin\z@ \linewidth\hsize
   \@setminipage}}%
 {\par\unskip\endMakeFramed%
 \at@end@of@kframe}
\makeatother

\definecolor{shadecolor}{rgb}{.97, .97, .97}
\definecolor{messagecolor}{rgb}{0, 0, 0}
\definecolor{warningcolor}{rgb}{1, 0, 1}
\definecolor{errorcolor}{rgb}{1, 0, 0}
\newenvironment{knitrout}{}{} % an empty environment to be redefined in TeX

\usepackage{alltt}
\usepackage[utf8]{inputenc}
\usepackage[UKenglish]{babel}
\usepackage{ragged2e}%pour justifier le text, après il suffit de tapper \justifying avant le paragraphe
\setbeamertemplate{navigation symbols}{}%no nav symbols
\usetheme[secheader]{Madrid}%

\title{Using data in R bal bal}
\author[Alexandre Courtiol]{Alexandre Courtiol}
\institute[IZW]{Leibniz Institute of Zoo and Wildlife Research}%
\date[June 2018]{\small June 2018}%
\IfFileExists{upquote.sty}{\usepackage{upquote}}{}
\begin{document}
\setlength{\topsep}{1pt}%space between input and output


\maketitle


\begin{frame}[containsverbatim]{Changing the order of levels of a factor}


\begin{columns}
\column{0.2\linewidth}
\begin{center}
You have:
\begin{knitrout}\scriptsize
\definecolor{shadecolor}{rgb}{0.969, 0.969, 0.969}\color{fgcolor}\begin{kframe}
\begin{alltt}
\hlstd{my_factor1}
\end{alltt}
\begin{verbatim}
## [1] A A B B C
## Levels: A B C
\end{verbatim}
\end{kframe}
\end{knitrout}
\end{center}

\column{0.2\linewidth}
\begin{center}
You want:
\begin{knitrout}\scriptsize
\definecolor{shadecolor}{rgb}{0.969, 0.969, 0.969}\color{fgcolor}\begin{kframe}
\begin{alltt}
\hlstd{my_factor2}
\end{alltt}
\begin{verbatim}
## [1] A A B B C
## Levels: C B A
\end{verbatim}
\end{kframe}
\end{knitrout}
\end{center}
\end{columns}

\begin{columns}
\column{0.5\linewidth}

\begin{center}
You do:
\begin{knitrout}\scriptsize
\definecolor{shadecolor}{rgb}{0.969, 0.969, 0.969}\color{fgcolor}\begin{kframe}
\begin{alltt}
\hlcom{## Using base:}
\hlstd{my_factor2} \hlkwb{<-} \hlkwd{factor}\hlstd{(my_factor1,} \hlkwd{levels}\hlstd{(my_factor1)[}\hlkwd{c}\hlstd{(}\hlnum{3}\hlstd{,} \hlnum{2}\hlstd{,} \hlnum{1}\hlstd{)])}
\hlstd{my_factor2}
\end{alltt}
\begin{verbatim}
## [1] A A B B C
## Levels: C B A
\end{verbatim}
\end{kframe}
\end{knitrout}

\end{center}
\end{columns}
\vfill

Note:

the order of levels influences the output of linear models and plotting functions (e.g. order in the legend of a ggplot) \dots
\end{frame}

\begin{frame}[containsverbatim]{Changing the levels of a factor}


\begin{columns}
\column{0.2\linewidth}
\begin{center}
You have:
\begin{knitrout}\scriptsize
\definecolor{shadecolor}{rgb}{0.969, 0.969, 0.969}\color{fgcolor}\begin{kframe}
\begin{alltt}
\hlstd{my_factor1}
\end{alltt}
\begin{verbatim}
## [1] A A B B C
## Levels: A B C
\end{verbatim}
\end{kframe}
\end{knitrout}
\end{center}

\column{0.2\linewidth}
\begin{center}
You want:
\begin{knitrout}\scriptsize
\definecolor{shadecolor}{rgb}{0.969, 0.969, 0.969}\color{fgcolor}\begin{kframe}
\begin{alltt}
\hlstd{my_factor2}
\end{alltt}
\begin{verbatim}
## [1] A A A A D
## Levels: A D
\end{verbatim}
\end{kframe}
\end{knitrout}
\end{center}
\end{columns}

\begin{columns}
\column{0.5\linewidth}

\begin{center}
You do:
\begin{knitrout}\scriptsize
\definecolor{shadecolor}{rgb}{0.969, 0.969, 0.969}\color{fgcolor}\begin{kframe}
\begin{alltt}
\hlcom{## Using base:}
\hlkwd{levels}\hlstd{(my_factor1)}
\end{alltt}
\begin{verbatim}
## [1] "A" "B" "C"
\end{verbatim}
\begin{alltt}
\hlstd{my_factor2} \hlkwb{<-} \hlstd{my_factor1}
\hlkwd{levels}\hlstd{(my_factor2)} \hlkwb{<-} \hlkwd{c}\hlstd{(}\hlstr{"A"}\hlstd{,} \hlstr{"A"}\hlstd{,} \hlstr{"D"}\hlstd{)} \hlcom{## in same order!}
\hlstd{my_factor2}
\end{alltt}
\begin{verbatim}
## [1] A A A A D
## Levels: A D
\end{verbatim}
\end{kframe}
\end{knitrout}

\begin{knitrout}\scriptsize
\definecolor{shadecolor}{rgb}{0.969, 0.969, 0.969}\color{fgcolor}\begin{kframe}
\begin{alltt}
\hlcom{## Using dplyr:}
\hlstd{my_factor2} \hlkwb{<-} \hlkwd{recode}\hlstd{(my_factor1,} \hlkwc{A} \hlstd{=} \hlstr{"A"}\hlstd{,} \hlkwc{B} \hlstd{=} \hlstr{"A"}\hlstd{,} \hlkwc{C} \hlstd{=} \hlstr{"D"}\hlstd{)}
\hlstd{my_factor2}
\end{alltt}
\begin{verbatim}
## [1] A A A A D
## Levels: A D
\end{verbatim}
\end{kframe}
\end{knitrout}
\end{center}
\end{columns}

\end{frame}

\end{document}
